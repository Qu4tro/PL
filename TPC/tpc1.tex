\documentclass{article}
\usepackage[utf8]{inputenc}

\title{tpc1}
\author{Xavier Francisco}
\date{March 2015}

\begin{document}


%---------------------------------------------------------------------------
\section{Expressões Regulares e Autómatos}
Considere as seguintes ERs:

    $e1\ =\ a\ b^{+}\ +\ a\ (c\ +\ d)^{+}\ b$
    
    $e2\ =\ c\ b^{+}\ +\ a\ b^{+}\ c$
    
    $e3\ =\ c\ (\varepsilon\ +\ a)\ b^{+}\ c$


Responda, então, às seguintes questões:
\begin{description}
  \item[a)] usando a respectiva \emph{cadeia de derivação}, diga se a frase "\texttt{acdb}" pertence àlinguagem gerada por $e1$.
  
      $e1\ =\ a\ b^{+}\ +\ a\ (c\ +\ d)^{+}\ b$
      
      $e1\ = a\ (c\ +\ d)^{+}\ b$
      
      $e1\ = a\ c\  (c\ +\ d)^{*}\ b$
      
      $e1\ = a\ c\ d\ b$
      
    Sim, a frase "\texttt{acdb}" pertence à linguagem gerada por $e1$.

  
  \item[b)] construa informalmente o Autómato Determinista equivalente a $e1$.
  \\
  \\
  \\
  \\
  \\
  \\
  \\
  \\
  \\
  \\
  \\
  \\
  \\
  \item[c)] diga, justificando, se $e2$ e $e3$  são equivalentes.
  
    $e3\ =\ c\ (\varepsilon\ +\ a)\ b^{+}\ c$
  
    $e3\ = (c\ +\ ca)\ b^{+}\ c$
  
    $e3\ = c\ b^{+}\ c +\ c\ a\ b^{+}\ c$  
    
    ou  
    
    $e3\ = c\ b^{+}\ c +\ c\ b^{+}\ c$  
    
    e2 não é equivalente a e3.
  
  \item[d)] modifique $e2$ de modo a permitir ter ainda frases: só com 1 $c$ ou só com $b$'s mas não admitir a fase nula $\varepsilon$.
  
    $e2\ =\ c\ b^{+}\ +\ a\ b^{+}\ c$
  
    $e2\ =\ (c\ +\ b^{+})\ +\ (c\ b^{+}\ +\ a\ b^{+}\ c)$
  
  
  
\end{description}

%---------------------------------------------------------------------------


\end{document}
